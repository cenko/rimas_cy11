%%%%%%%%%%%%%%%%%%%%%%%%%%%%%%%%%%%%%%%%%%%%%%%%%%%%%%%%%%%%%%%%%%%%%%%%%%%%%%%%%%
% NAME:		swift_scijust_template.tex
% LANGUAGE:	LaTeX
% AUTHOR:	Eleonora Troja, eleonora.troja@nasa.gov
% CREATED:	2014-08-01
% MODIFIED:	2014-08-01
%%%%%%%%%%%%%%%%%%%%%%%%%%%%%%%%%%%%%%%%%%%%%%%%%%%%%%%%%%%%%%%%%%%%%%%%%%%%%%%%%%
%% LaTeX template for the science justification & technical feasibility
%% to be submitted as part of a Swift Guest Investigator proposal.
%%
%% Swift Cycle 11
%% Deadline: 25 September 2014, 4.30pm EDT
%%
%%%%%%%%%%%%%%%%%%%%%%%%%%%%%%%%%%%%%%%%%%%%%%
%%%%% Default format: 11pt single column %%%%%
%%%%%%%%%%%%%%%%%%%%%%%%%%%%%%%%%%%%%%%%%%%%%%

\documentclass[letterpaper,11pt]{article}

%%%%%%%%%%%%%%%%%%%%%%%%%%%%%%%%%%%%
%%%%% Default font, two-column %%%%%
%%%%%%%%%%%%%%%%%%%%%%%%%%%%%%%%%%%%

%\documentclass[letterpaper,11pt,twocolumn]{article

%%%%%%%%%%%%%%%%%%%%%%%%%%%%%
%%%%% Included packages %%%%%
%%%%%%%%%%%%%%%%%%%%%%%%%%%%%

\usepackage{graphics,graphicx}
\usepackage{psfig}
\usepackage{times}

%%%%%%%%%%%%%%%%%%%%%%%%%%%%%%%%%%%%%%%%%%%%%%%%%
%%%%% Page dimensions                       %%%%%
%%%%% DO NOT CHANGE THE FOLLOWING 9 LINES!  %%%%%
%%%%%%%%%%%%%%%%%%%%%%%%%%%%%%%%%%%%%%%%%%%%%%%%%

\setlength{\textwidth}{7in} 
\setlength{\textheight}{9.5in}
\setlength{\topmargin}{-0.2in} 
\setlength{\oddsidemargin}{-0.2in}
\setlength{\evensidemargin}{-0.2in} 
\setlength{\headheight}{0in}
\setlength{\headsep}{0in} 
\setlength{\hoffset}{0in}
\setlength{\voffset}{0in}


%%%%%%%%%%%%%%%%%%%%%%%%%%%%%%%%%%
%%%%% Section heading format %%%%%
%%%%%%%%%%%%%%%%%%%%%%%%%%%%%%%%%%

\makeatletter
\renewcommand{\section}{\@startsection%
{section}{1}{0mm}{-\baselineskip}%
{0.5\baselineskip}{\normalfont\Large\bfseries}}%
\makeatother

%%%%%%%%%%%%%%%%%%%%%%%%%%%%%%%%%%%%%%%%%%%%%%%%%%%%%%%%%%%%%%%%%%%%%%%%%%%%%%%%%%
%  NOTES:
% 
%  - THE SCIENTIFIC JUSTIFICATION MUST NOT EXCEED 4 PAGES!
%     (The only exception are proposals in the "high redshihft GRB" category which can have up to 6 pages
%
%  - THE "BUDGET NARRATIVE" MUST BE LESS THAN 1 PAGE AND DOES  NOT  COUNT TOWARD THE ABOVE PAGE LIMIT
%
%  - IT IS STRONGLY RECOMMENDED TO USE A FONT SIZE OF 11pt OR LARGER.
%
%  - Do NOT include a CV, current and pending support, or amny other supporting information 
%
%  - THE SCIENTIFIC JUSTIFICATION MUST BE SUBMITTED AS A PDF  FILE:
%
%  latex swift_scijust_template.tex
%  dvips swift_scijust_template -o swift_scijust_template.ps
%  ps2pdf swift_scijust_template.ps swift_scijust_template.pdf
%
%%%%%%%%%%%%%%%%%%%%%%%%%%%%%%%%%%%%%%%%%%%%%%%%%%%%%%%%%%%%%%%%%%%%%%%%%%%%%%%%%%

\begin{document}
\pagestyle{plain}
\pagenumbering{arabic}

\begin{center} 
\bfseries\uppercase{Title of Your Proposal}
\end{center}
\vspace{-0.3cm}
\centerline{\bf PI: {A. Author}}
 
\noindent {\bf 1. Abstract}
\smallskip\\
Your concise abstract goes here. Please use the same abstract as in the Remote Proposal System (RPS) form. \\

\noindent {\bf 2. Description of the Proposed Program}
\smallskip\\
\noindent {\it A) Scientific Rationale:}
\smallskip\\
Give the scientific background of your proposed study.

\noindent {\it B) Immediate Objective:}
\smallskip\\
State what you are proposing to observe and what the goals of the proposed study are. \\

\noindent {\bf 3. Justification of Requested Observing Time, Feasibility, and Visibility}
\smallskip\\
The requested observing time, intrument modes, and any constraints have to be explained in detail here. 
In particular, give the expected significance of detection, count rates and fluxes, along with errors 
and background estimates for each of the instruments that are relevant to your study. 

Make sure that you check for bright stars within the UVOT field of view that might prohibit UVOT observations.
If you chose certain UVOT filters, you need to give a strong justification for the filter choice(s). 
If not strongly justified, UVOT observations will be performed in the ``Filter of the Day'' mode to 
minimize UVOT filter wheel rotations. 

Proposers must clearly describe how their proposal capitalizes on the unique capabilities of Swift.

If more than one target is proposed, give the priority of each target in case only some of your targets are accepted.
If this is a monitoring program, give the cadence of observations and all time constraints. 
If this is a ToO proposal, explain and justify the trigger criteria and give a realistic estimate of the probability 
that the ToO will be triggered.

If this is a joint proposal, proposers must provide a full and comprehensive justification
of the NRAO portion of their program. This must include the choice of telescopes (VLA, VLBA, and/or GBT),
and the total estimated observing time.


\noindent {\bf 4. Justification of Duplication}
\smallskip\\
It is the responsibility of the proposers to check the proposed observations against 
the catalog of previously executed observations or accepted programs. 
If any duplication exists, it must be identified and justified in the proposal.\\

\noindent {\bf 5. Report on Previous Swift and Related Programs}
\smallskip\\
Here you can report on previous studies (with Swift or non-Swift) that are related to this proposal. \\
If you had a proposal accepted during the past 3 Swift Cycles (Cycle 8, 9, or 10), you MUST give results here. \\

\noindent {\bf 6. References}
\smallskip\\
{\small
Author1 A.\ et al., YYYY, Journal, Vol, page \\
}

\noindent {\bf 7. Budget Narrative}
\smallskip\\
If you request funds, give a description of how the funds will be used and the scale of workforce needed 
to carry out the investigation. Give sufficient details and actual numbers to allow us to judge your effort. 
The page limit of the Budget Narrative section is 1-page, which will NOT count toward the overall page limit of your proposal. 


%-----------------------------Figure Start------------------------------
\begin{figure}[htp!]
\begin{center}
\hbox{
%%uncomment the following line to include your fig1a.ps postscript file:
%\psfig{figure=fig1a.ps,height=4.0cm,width=8.0cm,angle=0
}
\hspace{0.5cm}
% un-comment the following line to include your fig1b.ps postscript file:
%\psfig{figure=fig1b.ps,height=4.0cm,width=7.0cm,angle=0}
}
\end{center}
\caption{\footnotesize
{{\it Left panel}: here you see...
{\it Right panel}: here you see...}}
\label{fig1}
\end{figure}
%-----------------------------Figure End--------------------------------

\end{document}
